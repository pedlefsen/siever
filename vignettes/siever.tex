\documentclass{article}
\setlength{\parskip}{0.75em}
%\VignetteIndexEntry{siever}
\setlength{\evensidemargin}{0in}
\setlength{\oddsidemargin}{0in}
\setlength{\textwidth}{6.5in}

\usepackage{Sweave}
\begin{document}
\Sconcordance{concordance:siever.tex:siever.Rnw:%
1 7 1 1 0 45 1 1 2 1 0 1 1 60 0 1 59 1 1}

\begin{center}
\Large
{\tt siever} Package Vignette
\normalsize
\end{center}

\noindent Comparative analysis of pathogen genomes infecting vaccine versus placebo recipients.

This package contains R routines to implement the categorical sieve
analysis methods described in Gilbert, Wu, Jobes (2008) and variants
thereof, some of which are described in deCamp and Edlefsen (2014). See below for methods that are currently implemented.

These methods are used to examine the effect of vaccination on
infection as a function of a genomic feature of the pathogen (such as
a sequence feature that makes an HIV-1 virus a target of
vaccine-induced immunity).  The methods implemented in this R package
support comparative analysis of categorical data (usually amino acids)
at one or multiple loci when the vaccine-targeted category is known
and used as a reference.  The methods compare weights reflecting some
distance measure between the vaccine-targeted category and the
category of each infecting pathogen across treatment groups, with
weights usually given by an amino acid substitution matrix.

The present version implements only the non-parametric pooled-variance
t test from the original article.  The statistic $Z_2^A(i)$ is given in equation 4 of
Gilbert, Wu, Jobes (2008).  We also implement extensions and variations of the method
to support multiple observed sequences per subject (using an arbitrary
user-supplied function to compute per-subject weights from
per-sequence weights) and grouped testing of multiple sites (using
another arbitrary user-supplied function to compute per-site-set
statistics from per-site t statistics).  With these extensions, the
package implements the Simplified Mismatch Bootstrap (SMMB) and
Expected GWJ (EGWJ) methods described in deCamp and Edlefsen (2014)
and k-mer and whole-gene variants.

CHANGES LIST
\begin{enumerate}
  \item{renamed weightopt to SieverParameters}
  \item{added test-stat accumulator to SieverParameters}
    
\end{enumerate}

Here is an example of the typical usage.

\begin{Schunk}
\begin{Sinput}
> library("seqinr") # TODO: USE biostrings
> library("siever")
> 
> 
> # resetAssumptions <- function() {
> #   ## options to be manipulated in later tests
> #   weight.matrix <<- NULL 
> #   site.sets.list <<- NULL
> #   use.f.test <<- FALSE 
> #   return.t.test.result <<- FALSE
> #   weights.across.sites.in.a.set.init <<- 0 
> #   weights.across.sites.in.a.set.accumulation.fn <<- sum
> #   vaccine.sequence.sets.list <<-  NULL
> #   placebo.sequence.sets.list <<- NULL 
> #   mimic.smmb <<-  FALSE 
> #   instead.return.weights <<- FALSE
> #   ## fixed background manipulations of assumptions
> #   if(mimic.smmb) {
> #      weights.across.sequences.in.a.set.accumulation.fn <<- sum
> #   } else {     
> #      weights.across.sequences.in.a.set.accumulation.fn <<- mean
> #   }     
> #   if(is.null(weight.matrix)) {
> #      acceptable.chars <<- 
> #         c( "A","C","D","E","F","G","H","I","K","L","M","N","P","Q","R","S","T","V","W","Y","-" )
> #   } else {
> #      acceptable.chars <<- setdiff( rownames( weight.matrix ), 'X' )
> #   }     
> #   
> # }
> # 
> # ## Here we run filterAcceptable.  Do we need to have an "acceptableChars" in weightopt?  It's used again (redundantly I think) in dataWeights.
> # seqAmtx   <- siever:::filterAcceptable(siever:::getSeqMtx("control_A.fasta"))
> # seqBmtx   <- siever:::filterAcceptable(siever:::getSeqMtx("control_B.fasta"))
> # seqHXBmtx <- siever:::filterAcceptable(siever:::getSeqMtx("control_HXB2.fasta"))
> # 
> # resetAssumptions()
> # 
> # data.mtx <- rbind(seqAmtx, seqBmtx)
> # vaccine.indices <- 1:nrow(seqAmtx)
> # test.type <- "tTest" 
> #   
> # wOpt <- 
> #    weightopt(weight.mtx = weight.matrix, 
> #              mimic.smmb = mimic.smmb, 
> #              site.sets.list= site.sets.list, 
> #              obs.vSeq.sets.list= vaccine.sequence.sets.list,
> #              obs.pSeq.sets.list= placebo.sequence.sets.list, 
> #              acceptable.chars = acceptable.chars, # TODO: USED ANYMORE?  REMOVE?
> #              within.sites.acc.fn = weights.across.sites.in.a.set.accumulation.fn,
> #              within.obs.acc.fn = weights.across.sequences.in.a.set.accumulation.fn
> #              )
> #   
> #   GWJsieve(data.mtx=data.mtx,
> #            vaccine.indices=vaccine.indices, 
> #            insert.char.vector=as.vector( seqHXBmtx ),
> #            test.type = test.type,
> #            perm.num = 0, # TODO: RENAME.  THIS IS num.perms
> #            wOpt=wOpt)
\end{Sinput}
\end{Schunk}

\end{document}
